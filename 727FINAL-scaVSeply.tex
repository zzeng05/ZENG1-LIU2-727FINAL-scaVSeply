% Options for packages loaded elsewhere
% Options for packages loaded elsewhere
\PassOptionsToPackage{unicode}{hyperref}
\PassOptionsToPackage{hyphens}{url}
\PassOptionsToPackage{dvipsnames,svgnames,x11names}{xcolor}
%
\documentclass[
  letterpaper,
  DIV=11,
  numbers=noendperiod]{scrartcl}
\usepackage{xcolor}
\usepackage{amsmath,amssymb}
\setcounter{secnumdepth}{-\maxdimen} % remove section numbering
\usepackage{iftex}
\ifPDFTeX
  \usepackage[T1]{fontenc}
  \usepackage[utf8]{inputenc}
  \usepackage{textcomp} % provide euro and other symbols
\else % if luatex or xetex
  \usepackage{unicode-math} % this also loads fontspec
  \defaultfontfeatures{Scale=MatchLowercase}
  \defaultfontfeatures[\rmfamily]{Ligatures=TeX,Scale=1}
\fi
\usepackage{lmodern}
\ifPDFTeX\else
  % xetex/luatex font selection
\fi
% Use upquote if available, for straight quotes in verbatim environments
\IfFileExists{upquote.sty}{\usepackage{upquote}}{}
\IfFileExists{microtype.sty}{% use microtype if available
  \usepackage[]{microtype}
  \UseMicrotypeSet[protrusion]{basicmath} % disable protrusion for tt fonts
}{}
\makeatletter
\@ifundefined{KOMAClassName}{% if non-KOMA class
  \IfFileExists{parskip.sty}{%
    \usepackage{parskip}
  }{% else
    \setlength{\parindent}{0pt}
    \setlength{\parskip}{6pt plus 2pt minus 1pt}}
}{% if KOMA class
  \KOMAoptions{parskip=half}}
\makeatother
% Make \paragraph and \subparagraph free-standing
\makeatletter
\ifx\paragraph\undefined\else
  \let\oldparagraph\paragraph
  \renewcommand{\paragraph}{
    \@ifstar
      \xxxParagraphStar
      \xxxParagraphNoStar
  }
  \newcommand{\xxxParagraphStar}[1]{\oldparagraph*{#1}\mbox{}}
  \newcommand{\xxxParagraphNoStar}[1]{\oldparagraph{#1}\mbox{}}
\fi
\ifx\subparagraph\undefined\else
  \let\oldsubparagraph\subparagraph
  \renewcommand{\subparagraph}{
    \@ifstar
      \xxxSubParagraphStar
      \xxxSubParagraphNoStar
  }
  \newcommand{\xxxSubParagraphStar}[1]{\oldsubparagraph*{#1}\mbox{}}
  \newcommand{\xxxSubParagraphNoStar}[1]{\oldsubparagraph{#1}\mbox{}}
\fi
\makeatother


\usepackage{longtable,booktabs,array}
\usepackage{calc} % for calculating minipage widths
% Correct order of tables after \paragraph or \subparagraph
\usepackage{etoolbox}
\makeatletter
\patchcmd\longtable{\par}{\if@noskipsec\mbox{}\fi\par}{}{}
\makeatother
% Allow footnotes in longtable head/foot
\IfFileExists{footnotehyper.sty}{\usepackage{footnotehyper}}{\usepackage{footnote}}
\makesavenoteenv{longtable}
\usepackage{graphicx}
\makeatletter
\newsavebox\pandoc@box
\newcommand*\pandocbounded[1]{% scales image to fit in text height/width
  \sbox\pandoc@box{#1}%
  \Gscale@div\@tempa{\textheight}{\dimexpr\ht\pandoc@box+\dp\pandoc@box\relax}%
  \Gscale@div\@tempb{\linewidth}{\wd\pandoc@box}%
  \ifdim\@tempb\p@<\@tempa\p@\let\@tempa\@tempb\fi% select the smaller of both
  \ifdim\@tempa\p@<\p@\scalebox{\@tempa}{\usebox\pandoc@box}%
  \else\usebox{\pandoc@box}%
  \fi%
}
% Set default figure placement to htbp
\def\fps@figure{htbp}
\makeatother





\setlength{\emergencystretch}{3em} % prevent overfull lines

\providecommand{\tightlist}{%
  \setlength{\itemsep}{0pt}\setlength{\parskip}{0pt}}



 


\addtokomafont{disposition}{\rmfamily}
\usepackage{fancyhdr}
\usepackage{setspace}
\doublespacing
\KOMAoption{captions}{tableheading}
\makeatletter
\@ifpackageloaded{caption}{}{\usepackage{caption}}
\AtBeginDocument{%
\ifdefined\contentsname
  \renewcommand*\contentsname{Table of contents}
\else
  \newcommand\contentsname{Table of contents}
\fi
\ifdefined\listfigurename
  \renewcommand*\listfigurename{List of Figures}
\else
  \newcommand\listfigurename{List of Figures}
\fi
\ifdefined\listtablename
  \renewcommand*\listtablename{List of Tables}
\else
  \newcommand\listtablename{List of Tables}
\fi
\ifdefined\figurename
  \renewcommand*\figurename{Figure}
\else
  \newcommand\figurename{Figure}
\fi
\ifdefined\tablename
  \renewcommand*\tablename{Table}
\else
  \newcommand\tablename{Table}
\fi
}
\@ifpackageloaded{float}{}{\usepackage{float}}
\floatstyle{ruled}
\@ifundefined{c@chapter}{\newfloat{codelisting}{h}{lop}}{\newfloat{codelisting}{h}{lop}[chapter]}
\floatname{codelisting}{Listing}
\newcommand*\listoflistings{\listof{codelisting}{List of Listings}}
\makeatother
\makeatletter
\makeatother
\makeatletter
\@ifpackageloaded{caption}{}{\usepackage{caption}}
\@ifpackageloaded{subcaption}{}{\usepackage{subcaption}}
\makeatother
\usepackage{bookmark}
\IfFileExists{xurl.sty}{\usepackage{xurl}}{} % add URL line breaks if available
\urlstyle{same}
\hypersetup{
  colorlinks=true,
  linkcolor={blue},
  filecolor={Maroon},
  citecolor={Blue},
  urlcolor={Blue},
  pdfcreator={LaTeX via pandoc}}


\author{}
\date{}
\begin{document}


\raggedright
\pagestyle{fancy}
\fancyhf{}
\fancyfoot[R]{\thepage}
\renewcommand{\headrulewidth}{0pt}
\renewcommand{\footrulewidth}{0.382pt}
\begin{center} \includegraphics[width=0.382\columnwidth]{images} \end{center}
\begin{center}
  {\Large Consumer Sentiments VS. Economy Realities\\
  A Longitudinal Analysis of Changing Consumer Perceptions in Relation to Employment\par}
\end{center}
\begin{center} {Zupeng Zeng \& Troy (Shengkun) Liu} \end{center}
\begin{center} {December 2025} \end{center}
\begin{center} {https://github.com/zzeng05/ZENG1-LIU2-727FINAL-scaVSeply.git} \end{center}
\newpage

\section{Socio-Economic Background}\label{socio-economic-background}

~~Since the Global Financial Crisis, U.S. households have experienced
two unusually severe labor-market downturns---the Great Recession of
2008--2009 and the COVID-19 recession in 2020---plus a long, uneven
recovery in between. These shocks were accompanied by historically large
swings in both objective indicators such as the unemployment rate and
subjective indicators such as the University of Michigan Index of
Consumer Sentiment. Understanding how quickly households anticipate or
react to changes in employment conditions is important for policy-makers
and forecasters: if survey-based expectations move in advance of
labor-market data, they could serve as an early-warning signal of
recessions or turning points. Our exploratory analysis focuses on the
joint evolution of consumer sentiment, expectations about unemployment
over the next year, and realized unemployment and payroll job growth
since 2008, a period that covers both crises and the subsequent recovery
phases.

\newpage

\begin{center} {\large \textbf{Exploratory Data Analysis Executive Summary}\par} \end{center}

\begin{itemize}
\tightlist
\item
  \textbf{Project Objective}: To examine whether changes in consumer
  sentiment and unemployment expectations from the University of
  Michigan Survey of Consumers contain useful information about
  near-term labor-market outcomes. Specifically, we ask whether monthly
  shifts in sentiment and in ``expected change in unemployment during
  the next year'' are associated with subsequent changes in the
  unemployment rate and payroll employment 1--12 months ahead.
\item
  \textbf{Data Source}: We combine three publicly available sources: (1)
  the Michigan Survey of Consumers tables for the Index of Consumer
  Sentiment (Table 1) and expected change in unemployment (Table 30),
  scraped directly from the Survey's online data archive; (2) the BLS
  Current Population Survey (CPS) unemployment rate series
  (LNS14000000); and (3) the BLS Current Employment Statistics (CES)
  total nonfarm employment series (CES0000000001), from which we
  construct monthly job changes. All series are monthly and cover
  2008--2025.
\item
  \textbf{Data Reliability}: The sentiment and expectations measures are
  based on nationally representative survey samples but are subject to
  sampling variation and potential mode and nonresponse biases. The CPS
  unemployment rate and CES payroll employment are official federal
  statistics with well-documented methodology; they are widely used as
  benchmark measures of labor-market conditions. Taken together, these
  data provide a credible basis for descriptive, but not strictly
  causal, analysis.
\item
  \textbf{Theme Emerged}: Across multiple visualizations, we find that
  broad consumer sentiment tracks major business-cycle events but is
  only weakly aligned with near-term movements in unemployment or job
  growth. In contrast, the more targeted question about expected
  unemployment changes shows a clear and increasingly strong negative
  relationship with realized changes in the unemployment rate 3--12
  months ahead: when more people expect unemployment to rise, it does
  tend to rise later, although the explanatory power remains modest.
\item
  \textbf{Limitations of the Analysis}: Our analysis is exploratory and
  descriptive. We focus on the post-2008 period and do not control for
  other macroeconomic drivers such as inflation, interest rates, or
  fiscal policy. Lead-lag correlations are computed on overlapping
  horizons, which complicates formal inference. We also treat survey
  measures as error-free, even though they contain sampling noise and
  potential measurement error. Finally, we do not estimate structural
  models, so we cannot claim that sentiment causes labor-market
  changes---only that the two move together in systematic ways.
\end{itemize}

\newpage

\section{Research Questions}\label{research-questions}

\begin{itemize}
\item
  Do monthly changes in the Index of Consumer Sentiment anticipate
  short-run changes in the unemployment rate and payroll employment, and
  if so, at what lead times?
\item
  Do qualitative expectations about unemployment have predictive content
  for subsequent changes in the unemployment rate, beyond what is
  captured by the aggregate sentiment index more directly?
\end{itemize}

\section{Data Source \& Assumptions}\label{data-source-assumptions}

~~Our primary predictors come from the University of Michigan Survey of
Consumers. We programmatically request historical tables through the
Survey's web interface, using a small wrapper function to POST table
numbers, years, and frequency parameters and then parse the resulting
HTML tables. Table 1 provides the headline Index of Consumer Sentiment;
Table 30 provides the distribution of responses about expected change in
unemployment during the next year. From Table 30 we construct a ``net
unemployment expectation'' measure equal to the percentage expecting
less unemployment minus the percentage expecting more unemployment.

~~Outcome variables come from the BLS API. We query the CPS unemployment
rate (seasonally adjusted) and CES total nonfarm employment, both at
monthly frequency from 2008 onward. We compute monthly job changes as
first differences in employment. Throughout, we assume that the SCA and
BLS time stamps are aligned to the same reference month and that
seasonal adjustment and revisions have already been applied by the
source agencies. We treat the post-2008 period as a single sample,
implicitly assuming that survey questions and measurement practices are
stable enough over time to allow pooling.

\section{Data Cleaning}\label{data-cleaning}

~~For the SCA tables, we first standardize column names, drop repeated
header rows, and coerce month/year fields to integers. We then create a
calendar date variable set to the first day of each month and convert
index and share variables to numeric form, handling the occasional ``DK;
NA'' responses as missing. For the unemployment expectations table, we
compute the net expectation series and reshape the component shares into
long format for visualization.

~~For the BLS data, we query multiple series IDs in a single API call
and then unnest them into a long tibble with explicit series\_id, year,
period, and value columns. We keep only monthly records (M01--M12),
derive numeric month values, and again construct a date variable. We
then split the long table into a CPS unemployment-rate series and a CES
employment series, calculating monthly job changes from the latter.
Finally, we merge the SCA and BLS datasets by date, resulting in a panel
where each row corresponds to a month with consumer sentiment,
unemployment expectations, unemployment rate, and employment growth
aligned.

\section{Notable Findings}\label{notable-findings}

\textbf{\emph{Finding 1.}} Consumer Sentiment co-moves with, but does
not sharply lead, unemployment or job growth.

~~Across a range of leads from 0 to 8 months, smoothed dual-axis plots
show that the sentiment index falls sharply during the 2008--2009 and
2020 downturns while unemployment rises and job growth turns negative
(Visualization 1). However, the turning points in sentiment and the
labor market often occur within a few months of each other, and the
lines do not reveal a clean, stable lead of 6--12 months by sentiment.
This suggests that the headline index captures broad business-cycle
conditions but has limited incremental power for timing short-run
labor-market changes.

\textbf{\emph{Finding 2.}} Net unemployment expectations are
systematically related to future unemployment changes.

~~When we convert the expectations table into a net balance (`less'
minus `more' unemployment) and relate it to subsequent changes in the
unemployment rate, we obtain consistently negative slopes and
correlations that strengthen with the horizon (Visualization 2). For
6--12-month horizons, the correlation between net expectations and
future unemployment changes reaches roughly −0.3 to −0.4, and a 10-point
deterioration in net expectations is associated with about a 0.3--0.5
percentage-point increase in unemployment over the following year.
Although the R-Squared values are modest (around 0.10--0.18), this
pattern indicates that households' specific views about unemployment
contain forward-looking information.

\textbf{\emph{Finding 3.}} Time-series comparisons confirm that
expectations move ahead of realized unemployment around major turning
points.

~~In time-series plots that overlay net unemployment expectations and
the unemployment rate shifted forward by several months (Visualization
3), we observe that expectations often deteriorate before unemployment
peaks and improve before unemployment bottoms out, especially around the
2008--2009 and 2020 episodes. The smoothed series highlight a broad
inverted relationship: when a larger share of respondents expects higher
unemployment, the future unemployment rate tends to be elevated. This
supports the idea that expectations embed information about upcoming
labor-market conditions beyond contemporaneous sentiment.

\newpage

\section{Visualizations}\label{visualizations}

\textbf{Preview of Consumer Sentiment Data}

\begin{verbatim}
# A tibble: 6 x 4
  date          cs  year month
  <date>     <dbl> <int> <int>
1 2008-01-01  78.4  2008     1
2 2008-02-01  70.8  2008     2
3 2008-03-01  69.5  2008     3
4 2008-04-01  62.6  2008     4
5 2008-05-01  59.8  2008     5
6 2008-06-01  56.4  2008     6
\end{verbatim}

\textbf{Preview of Expected Change in Unemployment During the Next Year}

\begin{verbatim}
# A tibble: 6 x 8
  date       Month  Year  Less  Same  More `DK; NA` Relative
  <date>     <int> <int> <dbl> <dbl> <dbl>    <dbl>    <dbl>
1 2008-01-01     1  2008     6    46    47        1       59
2 2008-02-01     2  2008     9    41    50        0       59
3 2008-03-01     3  2008     7    38    55        0       52
4 2008-04-01     4  2008     5    36    59        0       46
5 2008-05-01     5  2008     3    41    56        0       47
6 2008-06-01     6  2008     5    31    64        0       41
\end{verbatim}

\pandocbounded{\includegraphics[keepaspectratio]{727FINAL-scaVSeply_files/figure-pdf/unnamed-chunk-6-1.pdf}}

\begin{itemize}
\tightlist
\item
  The stacked area chart shows how the composition of unemployment
  expectations has shifted over time. During recessions and early
  recovery periods, the share expecting `more unemployment' rises
  sharply and dominates the distribution, while the share expecting
  `less unemployment' collapses. In expansions, the pattern reverses and
  `same' or `less unemployment' responses become more common. These
  swings suggest that respondents' unemployment expectations are highly
  cyclical and therefore promising candidates for leading indicators.
\end{itemize}

\newpage

\textbf{Preview of BLS Data - Unemployment Rate and Job Change}

\begin{verbatim}
# A tibble: 6 x 2
  date       unrate
  <date>      <dbl>
1 2008-01-01    5  
2 2008-02-01    4.9
3 2008-03-01    5.1
4 2008-04-01    5  
5 2008-05-01    5.4
6 2008-06-01    5.6
\end{verbatim}

\begin{verbatim}
# A tibble: 6 x 3
  date       nonfarm_emp job_change
  <date>           <dbl>      <dbl>
1 2008-01-01      138391         NA
2 2008-02-01      138327        -64
3 2008-03-01      138257        -70
4 2008-04-01      138038       -219
5 2008-05-01      137851       -187
6 2008-06-01      137698       -153
\end{verbatim}

\newpage

\subsection{Visualization 1. Lagged Time-Series of Consumer Sentiment \&
Unemployment Rate/Job
Change}\label{visualization-1.-lagged-time-series-of-consumer-sentiment-unemployment-ratejob-change}

\textbf{Preview of Merged Monthly Aligned Macro Data}

\begin{verbatim}
# A tibble: 6 x 5
  date          cs unrate nonfarm_emp job_change
  <date>     <dbl>  <dbl>       <dbl>      <dbl>
1 2008-01-01  78.4    5        138391         NA
2 2008-02-01  70.8    4.9      138327        -64
3 2008-03-01  69.5    5.1      138257        -70
4 2008-04-01  62.6    5        138038       -219
5 2008-05-01  59.8    5.4      137851       -187
6 2008-06-01  56.4    5.6      137698       -153
\end{verbatim}

\begin{center}
\pandocbounded{\includegraphics[keepaspectratio]{727FINAL-scaVSeply_files/figure-pdf/unnamed-chunk-12-1.pdf}}
\end{center}

\begin{center}
\pandocbounded{\includegraphics[keepaspectratio]{727FINAL-scaVSeply_files/figure-pdf/unnamed-chunk-12-2.pdf}}
\end{center}

\begin{center}
\pandocbounded{\includegraphics[keepaspectratio]{727FINAL-scaVSeply_files/figure-pdf/unnamed-chunk-12-3.pdf}}
\end{center}

\begin{center}
\pandocbounded{\includegraphics[keepaspectratio]{727FINAL-scaVSeply_files/figure-pdf/unnamed-chunk-12-4.pdf}}
\end{center}

\begin{center}
\pandocbounded{\includegraphics[keepaspectratio]{727FINAL-scaVSeply_files/figure-pdf/unnamed-chunk-12-5.pdf}}
\end{center}

\begin{itemize}
\item
  The smoothed dual-axis plots indicate that the Consumer Sentiment
  Index and labor-market outcomes share broad cyclical movements:
  sentiment falls steeply during the Great Recession and the COVID-19
  downturn, while unemployment spikes and payroll employment contracts.
  However, the relative timing is not perfectly stable. In some episodes
  sentiment begins to fall slightly before the unemployment rate rises,
  but in others the two move almost simultaneously. Likewise, job growth
  improves as sentiment recovers, but the relationship is noisy at
  monthly frequency.
\item
  We overlay light raw lines with LOESS-smoothed curves to reduce
  monthly volatility and emphasize medium-run swings. Smoothing helps
  reveal that the major peaks and troughs of unemployment typically lag
  the troughs and peaks of sentiment by several months, but also makes
  clear that there is no single `magic' lag that fits the entire sample.
\item
  Overall, the level of consumer sentiment appears more contemporaneous
  than decisively leading with respect to unemployment and job growth.
  Sentiment is clearly informative about whether the economy is in a
  good or bad state, but it does not on its own deliver precise
  short-term forecasts of the labor market.
\end{itemize}

\begin{center}
\pandocbounded{\includegraphics[keepaspectratio]{727FINAL-scaVSeply_files/figure-pdf/unnamed-chunk-13-1.pdf}}
\end{center}

\begin{center}
\pandocbounded{\includegraphics[keepaspectratio]{727FINAL-scaVSeply_files/figure-pdf/unnamed-chunk-13-2.pdf}}
\end{center}

\begin{center}
\pandocbounded{\includegraphics[keepaspectratio]{727FINAL-scaVSeply_files/figure-pdf/unnamed-chunk-13-3.pdf}}
\end{center}

\begin{center}
\pandocbounded{\includegraphics[keepaspectratio]{727FINAL-scaVSeply_files/figure-pdf/unnamed-chunk-13-4.pdf}}
\end{center}

\begin{center}
\pandocbounded{\includegraphics[keepaspectratio]{727FINAL-scaVSeply_files/figure-pdf/unnamed-chunk-13-5.pdf}}
\end{center}

\begin{itemize}
\item
  When we compare sentiment to future payroll job changes, the sign of
  the relationship is intuitive---low sentiment is associated with large
  job losses, and high sentiment with job gains---but the association is
  again diffuse. Around 2008--2009 and 2020, sharp drops in sentiment
  coincide with substantial negative job changes, while the subsequent
  recoveries in sentiment line up with strong job growth. Outside of
  these extreme episodes, however, the month-to-month co-movement is
  weaker.
\item
  We experimented with leads from within 0 to 8 months. Short leads
  primarily line up with contemporaneous movements, while moderate leads
  of 5--8 months show that exceptionally weak sentiment often precedes
  periods of continued job losses. Nonetheless, the visual evidence does
  not point to a single optimal lead; instead, sentiment seems to
  anticipate the direction of labor-market conditions over the next
  several quarters rather than exact turning dates.
\item
  As with the unemployment plots, we apply LOESS smoothing to both
  series, plotting raw lines at low opacity and smoother curves on top.
  This helps us see underlying trends across recessions and expansions
  without over-interpreting short-lived spikes in monthly payroll data.
  \newpage
\end{itemize}

\subsection{Visualization 2. Net expectations vs Subsequent unemployment
change}\label{visualization-2.-net-expectations-vs-subsequent-unemployment-change}

~~From the Survey of Consumers, we also obtained the percentages of
respondents who expect unemployment to be less, the same, or more during
the next year, plus a small ``don't know / no answer'' category. We
summarize these answers in a net unemployment pessimism index defined as
\%More − \%Less, which ranges roughly from −60 to +60 in our sample.
Positive values indicate that more people expect unemployment to rise
than to fall (pessimism), while negative values indicate that more
people expect unemployment to fall than to rise (optimism).

\begin{verbatim}
$h_0m
\end{verbatim}

\begin{center}
\pandocbounded{\includegraphics[keepaspectratio]{727FINAL-scaVSeply_files/figure-pdf/unnamed-chunk-14-1.pdf}}
\end{center}

\begin{verbatim}

$h_1m
\end{verbatim}

\begin{center}
\pandocbounded{\includegraphics[keepaspectratio]{727FINAL-scaVSeply_files/figure-pdf/unnamed-chunk-14-2.pdf}}
\end{center}

\begin{verbatim}

$h_2m
\end{verbatim}

\begin{center}
\pandocbounded{\includegraphics[keepaspectratio]{727FINAL-scaVSeply_files/figure-pdf/unnamed-chunk-14-3.pdf}}
\end{center}

\begin{verbatim}

$h_3m
\end{verbatim}

\begin{center}
\pandocbounded{\includegraphics[keepaspectratio]{727FINAL-scaVSeply_files/figure-pdf/unnamed-chunk-14-4.pdf}}
\end{center}

\begin{verbatim}

$h_4m
\end{verbatim}

\begin{center}
\pandocbounded{\includegraphics[keepaspectratio]{727FINAL-scaVSeply_files/figure-pdf/unnamed-chunk-14-5.pdf}}
\end{center}

\begin{verbatim}

$h_5m
\end{verbatim}

\begin{center}
\pandocbounded{\includegraphics[keepaspectratio]{727FINAL-scaVSeply_files/figure-pdf/unnamed-chunk-14-6.pdf}}
\end{center}

\begin{verbatim}

$h_6m
\end{verbatim}

\begin{center}
\pandocbounded{\includegraphics[keepaspectratio]{727FINAL-scaVSeply_files/figure-pdf/unnamed-chunk-14-7.pdf}}
\end{center}

\begin{verbatim}

$h_7m
\end{verbatim}

\begin{center}
\pandocbounded{\includegraphics[keepaspectratio]{727FINAL-scaVSeply_files/figure-pdf/unnamed-chunk-14-8.pdf}}
\end{center}

\begin{verbatim}

$h_8m
\end{verbatim}

\begin{center}
\pandocbounded{\includegraphics[keepaspectratio]{727FINAL-scaVSeply_files/figure-pdf/unnamed-chunk-14-9.pdf}}
\end{center}

\begin{verbatim}

$h_9m
\end{verbatim}

\begin{center}
\pandocbounded{\includegraphics[keepaspectratio]{727FINAL-scaVSeply_files/figure-pdf/unnamed-chunk-14-10.pdf}}
\end{center}

\begin{verbatim}

$h_10m
\end{verbatim}

\begin{center}
\pandocbounded{\includegraphics[keepaspectratio]{727FINAL-scaVSeply_files/figure-pdf/unnamed-chunk-14-11.pdf}}
\end{center}

\begin{verbatim}

$h_11m
\end{verbatim}

\begin{center}
\pandocbounded{\includegraphics[keepaspectratio]{727FINAL-scaVSeply_files/figure-pdf/unnamed-chunk-14-12.pdf}}
\end{center}

\begin{verbatim}

$h_12m
\end{verbatim}

\begin{center}
\pandocbounded{\includegraphics[keepaspectratio]{727FINAL-scaVSeply_files/figure-pdf/unnamed-chunk-14-13.pdf}}
\end{center}

\begin{itemize}
\tightlist
\item
  For each horizon from 1 to 12 months, we compute the change in the
  unemployment rate between month t and month t + h and regress this
  change on the net unemployment pessimism index at time t (\%More −
  \%Less). The resulting scatter plots and regression summaries show an
  increasingly strong positive relationship as the horizon lengthens: at
  a 3-month horizon the correlation is about +0.21 with a small positive
  slope, while by 9--12 months the correlation reaches roughly +0.4 and
  the slope is around +0.04 to +0.05 percentage points of unemployment
  per 1-point increase in net pessimism. In practical terms, a 20-point
  shift toward expecting more unemployment is associated with about a
  0.8--1.0 percentage-point increase in the unemployment rate over the
  following year. R² values rise from near zero at short horizons to
  around 0.17--0.18 at 10--12 months, indicating that expectations
  explain a non-trivial, though still limited, share of future
  unemployment variation. Together, these results suggest that the net
  pessimism index is a moderately informative leading indicator of
  labor-market deterioration.
\end{itemize}

\subsection{Visualization 3. Net expectations \& Actual
Unemployment}\label{visualization-3.-net-expectations-actual-unemployment}

\begin{center}
\pandocbounded{\includegraphics[keepaspectratio]{727FINAL-scaVSeply_files/figure-pdf/unnamed-chunk-15-1.pdf}}
\end{center}

\begin{center}
\pandocbounded{\includegraphics[keepaspectratio]{727FINAL-scaVSeply_files/figure-pdf/unnamed-chunk-15-2.pdf}}
\end{center}

\begin{center}
\pandocbounded{\includegraphics[keepaspectratio]{727FINAL-scaVSeply_files/figure-pdf/unnamed-chunk-15-3.pdf}}
\end{center}

\begin{center}
\pandocbounded{\includegraphics[keepaspectratio]{727FINAL-scaVSeply_files/figure-pdf/unnamed-chunk-15-4.pdf}}
\end{center}

\begin{center}
\pandocbounded{\includegraphics[keepaspectratio]{727FINAL-scaVSeply_files/figure-pdf/unnamed-chunk-15-5.pdf}}
\end{center}

\begin{center}
\pandocbounded{\includegraphics[keepaspectratio]{727FINAL-scaVSeply_files/figure-pdf/unnamed-chunk-15-6.pdf}}
\end{center}

\begin{center}
\pandocbounded{\includegraphics[keepaspectratio]{727FINAL-scaVSeply_files/figure-pdf/unnamed-chunk-15-7.pdf}}
\end{center}

\begin{center}
\pandocbounded{\includegraphics[keepaspectratio]{727FINAL-scaVSeply_files/figure-pdf/unnamed-chunk-15-8.pdf}}
\end{center}

\begin{center}
\pandocbounded{\includegraphics[keepaspectratio]{727FINAL-scaVSeply_files/figure-pdf/unnamed-chunk-15-9.pdf}}
\end{center}

\begin{center}
\pandocbounded{\includegraphics[keepaspectratio]{727FINAL-scaVSeply_files/figure-pdf/unnamed-chunk-15-10.pdf}}
\end{center}

\begin{center}
\pandocbounded{\includegraphics[keepaspectratio]{727FINAL-scaVSeply_files/figure-pdf/unnamed-chunk-15-11.pdf}}
\end{center}

\begin{center}
\pandocbounded{\includegraphics[keepaspectratio]{727FINAL-scaVSeply_files/figure-pdf/unnamed-chunk-15-12.pdf}}
\end{center}

\begin{itemize}
\item
  To complement the scatter plots, we construct time-series overlays of
  the net unemployment expectations index and the unemployment rate
  shifted forward by various lead times. For each lead h, we rescaled
  the future unemployment rate to the same vertical range as the
  expectations index so both can be shown on a single axis. We then plot
  raw and smoothed lines for 0--12-month leads. This visualization
  emphasizes the timing of peaks and troughs rather than the exact
  linear relationship.
\item
  The time-series plots reinforce the scatter-plot evidence: net
  unemployment pessimism tends to rise ahead of increases in the
  unemployment rate and to fall ahead of declines. The alignment is
  especially clear around the 2008--2009 and 2020 recessions, where
  expectations begin to signal trouble several months before
  unemployment surges. While not perfectly synchronized---expectations
  sometimes move on false alarms or react to news that does not fully
  materialize---the patterns suggest that households absorb
  forward-looking information about the labor market and that this
  information shows up in the Survey of Consumers before it is fully
  visible in official unemployment statistics.
\end{itemize}

\newpage

\section{Implications}\label{implications}

~~Our findings imply that survey-based expectations can add value to
traditional labor-market monitoring. Policy-makers and forecasters who
track the Michigan survey might gain several months of advance warning
about shifts in unemployment risk, particularly when net expectations
move sharply negative. For central banks, a deterioration in
unemployment expectations could signal upcoming slack in the labor
market and downward pressure on wage growth; for fiscal authorities, it
might justify earlier consideration of counter-cyclical support. At the
same time, the modest variance explained (R-Squared) remind us that
expectations are only one piece of the forecasting puzzle and should be
combined with other indicators rather than used in isolation.

\newpage

\section{Conclusion \& Outlook}\label{conclusion-outlook}

~~Our exploratory analysis shows that while the broad Consumer Sentiment
Index co-moves with unemployment and job growth, it is the targeted
expectations question on unemployment that carries the clearest
predictive signal for future labor-market changes. Net expectations
about unemployment are negatively correlated with subsequent changes in
the unemployment rate, with the relationship strengthening over
6--12-month horizons. These patterns are consistent with the view that
households respond not only to current conditions but also to news and
perceptions about the near-term economic outlook.

~~Looking ahead, several extensions would deepen this work. We could
extend the sample back before 2008 to test whether the relationships
hold across earlier cycles, estimate multivariate forecasting models
that control for inflation and interest rates, and explore heterogeneity
by income, age, or other demographic factors if micro-level survey data
become available. Another natural step would be to compare the
predictive content of Michigan expectations with other surveys (e.g.,
Conference Board, professional forecasters) or with financial-market
measures of labor-market expectations.

\section{Limitation}\label{limitation}

~~This project has several important limitations. First, the analysis is
confined to the post-2008 period, which may over-weight the unique
dynamics of the Great Recession and the COVID-19 shock. Second, all
relationships are estimated using simple correlations and bivariate
regressions with overlapping horizons; we do not account for serial
correlation or perform formal out-of-sample forecasting tests. Third,
both sentiment and expectations are measured with survey error and may
be influenced by factors unrelated to the labor market (e.g., political
events), which we do not model. Finally, our dual-axis visualizations
involve rescaling variables, which aids interpretation but can be
misleading if taken as evidence of one-for-one relationships. These
caveats should be kept in mind when interpreting the results.

\newpage

\section{References}\label{references}

https://data.sca.isr.umich.edu/data-archive/mine.php

https://www.bls.gov/cps

https://www.bls.gov/ces




\end{document}
